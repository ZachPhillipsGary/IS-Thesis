	%%%%%%%%%%%%%%%%%%%%%%%%%%%%%%%%%%%%%%%%%%%%%%%%%%%%%%%%%%%%%%%%%%%%%%%%%%%%%%%%%%%%%%%%%%%%%%
	%
	%                                                       Example IS Template
	%
	% \documentclass{woosterthesis} must be at the beginning of every IS. Options are the same as
	% for the report class with some additional options, abstractonly, blacklinks, code, kaukecopyright, palatino, picins,
	% maple, index, verbatim, dropcaps, euler, gauss, alltt,  woolshort, colophon, woosterchicago, and
	% achemso. The kaukecopyright option will put the arch symbol with the word mark on the
	% copyright page. The woosterthesis class is based on the report class. One thing to note is that
	% the ``%'' symbol comments out all characters that follow it on the line.
	%%%%%%%%%%%%%%%%%%%%%%%%%%%%%%%%%%%%%%%%%%%%%%%%%%%%%%%%%%%%%%%%%%%%%%%%%%%%%%%%%%%%%%%%%%%%%%
	
	%%%%%%%%%%%%%%%%%%%%%%%%%%%%%%%%%%%%%%%%%%%%%%%%%%%%%%%%%%%%%%%%%%%%%%%%%%%%%%%%%%%%%%%%%%%%%%
	% use this declaration for a draft  version of your IS
	\documentclass[10pt,palatino,code,picins,kaukecopyright,openright,woolshort,dropcaps,verbatim,index,euler]{woosterthesis}
	%\documentclass[10pt,code,picins,kaukecopyright,openright,woolshort,dropcaps,verbatim,euler,index,colophon,blacklinks,twoside]{woosterthesis}
	% note that you can specify the woosterchicago option to use Chicago citation style and achemso to use the American Chemical Society citation format
	%
	%%%%%%%%%%%%%%%%%%%%%%%%%%%%%%%%%%%%%%%%%%%%%%%%%%%%%%%%%%%%%%%%%%%%%%%%%%%%%%%%%%%%%%%%%%%%%%
	% use this declaration for the print version of your IS
	%\documentclass[12pt,code,palatino,picins,blacklinks,kaukecopyright,openright,twoside]{woosterthesis} % probably what most students would use
	%
	%%%%%%%%%%%%%%%%%%%%%%%%%%%%%%%%%%%%%%%%%%%%%%%%%%%%%%%%%%%%%%%%%%%%%%%%%%%%%%%%%%%%%%%%%%%%%%
	% use this declaration for the CD version of your IS
	%\documentclass[12pt,code,palatino,picins,kaukecopyright,openright,twoside]{woosterthesis}
	%%%%%%%%%%%%%%%%%%%%%%%%%%%%%%%%%%%%%%%%%%%%%%%%%%%%%%%%%%%%%%%%%%%%%%%%%%%%%%%%%%%%%%%%%%%%%%
	
	%%%%%%%%%%%%%%%%%%%%%%%%%%%%%%%%%%%%%%%%%%%%%%%%%%%%%%%%%%%%%%%%%%%%%%%%%%%%%%%%%%%%%%%%%%%%%%
	%
	%                                                       Load Packages
	%
	%   To load packages in addition to the ones that are loaded by default, please place your
	%   usepackage commands in the packages.tex file in the styles folder.
	%
	%
	%%%%%%%%%%%%%%%%%%%%%%%%%%%%%%%%%%%%%%%%%%%%%%%%%%%%%%%%%%%%%%%%%%%%%%%%%%%%%%%%%%%%%%%%%%%%%%
	
	\input{styles/packages}
	
	%%%%%%%%%%%%%%%%%%%%%%%%%%%%%%%%%%%%%%%%%%%%%%%%%%%%%%%%%%%%%%%%%%%%%%%%%%%%%%%%%%%%%%%%%%%%%%
	%
	%                                                       Load Personal commands
	%                                                                    
	% There will be certain commands that you use frequently in the thesis. You can give these
	% commands new names which are easier for you to remember. You can also combine several
	% commands into a new command of your own. See The LaTeX Companion or Guide to LaTeX
	% for examples on defining your own commands. These are commands that I defined to cut
	% down on typing. You can enter your commands in the personal.tex file in the styles folder.
	%
	%%%%%%%%%%%%%%%%%%%%%%%%%%%%%%%%%%%%%%%%%%%%%%%%%%%%%%%%%%%%%%%%%%%%%%%%%%%%%%%%%%%%%%%%%%%%%%
	
	\input{styles/personal}
	
	%%%%%%%%%%%%%%%%%%%%%%%%%%%%%%%%%%%%%%%%%%%%%%%%%%%%%%%%%%%%%%%%%%%%%%%%%%%%%%%%%%%%%%%%%%%%%%
	%
	%                                                       Load Theorem formatting information
	%
	%  If you need to define an new theorem style or want to see what theorem like environments 
	%  are available please look at the theorems.tex file in the styles folder.
	%
	%%%%%%%%%%%%%%%%%%%%%%%%%%%%%%%%%%%%%%%%%%%%%%%%%%%%%%%%%%%%%%%%%%%%%%%%%%%%%%%%%%%%%%%%%%%%%%
	
	\input{styles/theorems}
	
	\setcounter{secnumdepth}{5}% controls the numbering of sections
	\setcounter{tocdepth}{6}% controls the number of levels in the Contents
	
	%%%%%%%%%%%%%%%%%%%%%%%%%%%%%%%%%%%%%%%%%%%%%%%%%%%%%%%%%%%%%%%%%%%%%%%%%%%%%%%%%%%%%%%%%%%%%%
	% This is where one enters the information about the thesis.
	%%%%%%%%%%%%%%%%%%%%%%%%%%%%%%%%%%%%%%%%%%%%%%%%%%%%%%%%%%%%%%%%%%%%%%%%%%%%%%%%%%%%%%%%%%%%%%
	
	
	\title{Almost Real: An investigation into the metaphysics of virtual reality.}
	\thesistype{Example Independent Study Thesis} % you should make this Independent Study Thesis
	\author{Zachary A Phillips-Gary}
	%\presentdegrees{Ph.D.} % you should comment this line
	\degreetoobtain{Bachelor of Arts In Computer Science and Philosophy}
	\presentschool{The College of Wooster}
	\academicprogram{Department of Math and Computer Science }
	\gradyear{2016}
	\advisor{Denise Byres, Ph.D}
	\secondadvisor{Elizabeth Schiltz, Ph.D}
	%\reader{Reader}
	\copyrighted   
	%\copyrightdate{}                  
	\makeindex % comment this line if you do not have an index
	
	%%%%%%%%%%%%%%%%%%%%%%%%%%%%%%%%%%%%%%%%%%%%%%%%%%%%%%%%%%%%%%%%%%%%%%%%%%%%%%%%%%%%%%%%%%%%%%
	% This is where the commands for the document begin. All \LaTeX{} documents must have a
	% \begin{document} text .... \end{document} structure.
	%%%%%%%%%%%%%%%%%%%%%%%%%%%%%%%%%%%%%%%%%%%%%%%%%%%%%%%%%%%%%%%%%%%%%%%%%%%%%%%%%%%%%%%%%%%%%%
	
	\begin{document}
	
	%%%%%%%%%%%%%%%%%%%%%%%%%%%%%%%%%%%%%%%%%%%%%%%%%%%%%%%%%%%%%%%%%%%%%%%%%%%%%%%%%%%%%%%%%%%%%%
	% The front matter includes acknowledgments, dedications, vitas, list of tables, list of figures,
	% copyright, abstract, title page, and contents.
	%%%%%%%%%%%%%%%%%%%%%%%%%%%%%%%%%%%%%%%%%%%%%%%%%%%%%%%%%%%%%%%%%%%%%%%%%%%%%%%%%%%%%%%%%%%%%%
	
	\frontmatter
	\maketitle
	\ClearShipoutPicture
	\clearpage\thispagestyle{empty}\null\clearpage
	\disscopyright 
	
	%%%%%%%%%%%%%%%%%%%%%%%%%%%%%%%%%%%%%%%%%%%%%%%%%%%%%%%%%%%%%%%%%%%%%%%%%%%%%%%%%%%%%%%%%%%%%%
	%                                                                                       
	%                                                       Abstract						
	%                                                                                       
	%%%%%%%%%%%%%%%%%%%%%%%%%%%%%%%%%%%%%%%%%%%%%%%%%%%%%%%%%%%%%%%%%%%%%%%%%%%%%%%%%%%%%%%%%%%%%%
	
	\begin{abstract}
	This interdisciplinary independent study thesis is divided into five chapters. The first
	chapter "On Virtual Realism" will outline the author's unique formulation of virtual
	realism, the ontological thesis that experiences in virtual reality are non-illusory, and
	that virtual objects truly exist in an equivalent sense to physical objects. Next,
	"Idealism and Virtual Reality" will describe the metaphysical implications of the
	idealism of British empiricist George Berkeley and the Yogacaran school of Mahayana
	Buddhism with respect to a virtual realist project. Although Berkeley and the
	Yogacaran philosopher Vasubandhu differ in several respects, both author’s projects
	are ultimately sympathetic to the aims of virtual realism. However, the Yogacaran
	idealism ultimately proves to be the superior choice for the virtual realist. The next
	chapter "Virtual physicalism" will posit a physicalist reply to the idealism of
	Vasubandhu and argue for a physicalist account of reality and virtual reality using
	Occam's razor as inspired by Saul Kripke's argumentation in his article Mad pain and
	Martian pain. Having established the merits of physicalist account of reality over the
	idealist standpoint, the remainder of this section will focus on how the metaphysical
	commitments of physicalism impact a virtual realist worldview. The fourth chapter
	will describe the implementation of an immersive virtual environment using
	physicalist principles and discuss how awareness of the virtual nature of a virtual
	reality environment impacts an agent's experience of said environment. Of specific
	interest to this project is how ideas from physicalist metaphysics can aid virtual reality
	designers in crafting non-player character experiences which do not fall victim to the
	uncanny valley phenomenon (Masahiro Mori's hypothesis that very human-like but
	slightly imperfect digital entities elicit feelings of eeriness and revulsion among some
	observers). The fifth chapter describes the methodology and results of a usability study
	which attempts to measure the effectiveness of these philosophical principles in
	helping designers remedy the uncanny valley phenomenon in their creations.
	\end{abstract}
	
	%%%%%%%%%%%%%%%%%%%%%%%%%%%%%%%%%%%%%%%%%%%%%%%%%%%%%%%%%%%%%%%%%%%%%%%%%%%%%%%%%%%%%%%%%%%%%%
	%                                                                                       
	%                                                       Dedications					
	%                                                                                       
	%%%%%%%%%%%%%%%%%%%%%%%%%%%%%%%%%%%%%%%%%%%%%%%%%%%%%%%%%%%%%%%%%%%%%%%%%%%%%%%%%%%%%%%%%%%%%%
	
	\dedication{This work is dedicated to the future generations of Wooster students.}
	
	
	%%%%%%%%%%%%%%%%%%%%%%%%%%%%%%%%%%%%%%%%%%%%%%%%%%%%%%%%%%%%%%%%%%%%%%%%%%%%%%%%%%%%%%%%%%%%%%
	%                                                                                       
	%                                                       Acknowledgments					
	%                                                                                       
	%%%%%%%%%%%%%%%%%%%%%%%%%%%%%%%%%%%%%%%%%%%%%%%%%%%%%%%%%%%%%%%%%%%%%%%%%%%%%%%%%%%%%%%%%%%%%%
	
	\begin{acknowl}  
	I would like to acknowledge Prof. Lowell Boone in the Physics Department for his suggestions and code.
	\end{acknowl}
	
	%%%%%%%%%%%%%%%%%%%%%%%%%%%%%%%%%%%%%%%%%%%%%%%%%%%%%%%%%%%%%%%%%%%%%%%%%%%%%%%%%%%%%%%%%%%%%%
	%                                                                                       
	%                                                       Vita					
	%                                                                                       
	%%%%%%%%%%%%%%%%%%%%%%%%%%%%%%%%%%%%%%%%%%%%%%%%%%%%%%%%%%%%%%%%%%%%%%%%%%%%%%%%%%%%%%%%%%%%%%
	
	\begin{vita} 
	% You talk about yourself and how you got to where you are now. There is a structured form for the Vita that can be used if you want, but I don't encourage it.
	
	%%%%%%%%%%%%%%%%%%%%%%%%%%%%%%%%%%%%%%%%%%%%%%%%%%%%%%%%%%%%%%%%%%%%%%%%%%%%%%%%%%%%%%%%%%%%%%
	% The list below is for a thesis that requires a more structured Vita such as a masters or Ph.D.
	%%%%%%%%%%%%%%%%%%%%%%%%%%%%%%%%%%%%%%%%%%%%%%%%%%%%%%%%%%%%%%%%%%%%%%%%%%%%%%%%%%%%%%%%%%%%%%
	
	%\begin{datelist}
	%\item[April 6, 1970]Born-Wooster, Ohio
	%\item[August 11, 1990]Chosen to present an undergraduate paper at the 75th meeting of the MAA, Columbus, Ohio
	%\item[August 1990--August 1991]President Wooster Student Chapter of the MAA, The College of Wooster, Wooster, Ohio
	%\item[August 1991--May 1992]Secretary Wooster Student Chapter of the MAA, The College of Wooster, Wooster, Ohio
	%\item[1992]\emph{Phi Beta Kappa} (on junior standing), The College of Wooster, Wooster, Ohio
	%\item[1992]Elizabeth Sidwell Wagner Prize in Mathematics, The College of Wooster
	%\item[1992]William H. Wilson Prize in Mathematics, The College of Wooster
	%\item[May 11, 1992]B.A., Mathematics, The College of Wooster
	%\item[1997]Finalist for Graduate Teaching Award, The Ohio State University, Columbus, Ohio
	%\item[June 21-25, 1998]Participant in the AMS-IMS-SIAM Summer Research Conferences: q-Series, Combinatorics, and Computer Algebra, Mt. Holyoke, Massachusetts
	%\item[October 1998--October 1999]Graduate student representative to The Ohio State University Department of Mathematics Graduate Studies Committee, Columbus, Ohio
	%\item[January 1999]q-series seminar address, The Ohio State University, Columbus, Ohio
	%\item[2000]Finalist for Departmental Teaching Award, The Ohio State University, Columbus, Ohio
	%\item[2000]Nominated for Graduate Teaching Award, The Ohio State University, Columbus, Ohio
	%\item[April 2000]Invited colloquium talk at The College of Wooster, Wooster, Ohio
	%\item[1992-- present]Graduate Teaching and Research Associate, The Ohio State University
	%\end{datelist}
	%
	%%%This is for any publications you might have.%%%%%
	
	\begin{publist}  
	\pubitem{\quad}
	\pubitem{\quad}
	\end{publist} 
	
	\begin{fieldsstudy} 
	    \majorfield{Computer Science and Philosophy}
	
	    \specialization{Human Computer Interactions, Metaphysics, Virtual Reality}
	    %\begin{studieslist}
	   %\studyitem{Abstract Algebra}{Hampton}
	   %\end{studieslist}
	  \end{fieldsstudy}
	\end{vita}
	
	%%%%%%%%%%%%%%%%%%%%%%%%%%%%%%%%%%%%%%%%%%%%%%%%%%%%%%%%%%%%%%%%%%%%%%%%%%%%%%%%%%%%%%%%%%%%%%
	% We now create the contents page and if necessary the list of figures and list of tables.
	%%%%%%%%%%%%%%%%%%%%%%%%%%%%%%%%%%%%%%%%%%%%%%%%%%%%%%%%%%%%%%%%%%%%%%%%%%%%%%%%%%%%%%%%%%%%%%
	
	
	\cleardoublepage
	\phantomsection
	\addcontentsline{toc}{chapter}{Contents}
	
	\tableofcontents
	\listoffigures %Use if you have a list of figures.
	\listoftables%Use if you have a list of tables.
	\lstlistoflistings% Use if you are using the code option
	
	%%%%%%%%%%%%%%%%%%%%%%%%%%%%%%%%%%%%%%%%%%%%%%%%%%%%%%%%%%%%%%%%%%%%%%%%%%%%%%%%%%%%%%%%%%%%%%
	
	\input{chapters/preface} % most theses do not have a preface so this should be commented
	
	%%%%%%%%%%%%%%%%%%%%%%%%%%%%%%%%%%%%%%%%%%%%%%%%%%%%%%%%%%%%%%%%%%%%%%%%%%%%%%%%%%%%%%%%%%%%%%
	\mainmatter
	
	%%%%%%%%%%%%%%%%%%%%%%%%%%%%%%%%%%%%%%%%%%%%%%%%%%%%%%%%%%%%%%%%%%%%%%%%%%%%%%%%%%%%%%%%%%%%%%
	%
	%                                                       Thesis Chapters
	%
	% This is where the main text of the thesis goes. I have written this template assuming that
	% each chapter is a separate file. You do not have to do this but it makes things easier to find
	% for editing. You can use the sample chapters to help you figure out how to type things into
	% your thesis. To include a chapter just use the \include{chaptername} command. Chapters are
	% included in the order listed.
	%
	%%%%%%%%%%%%%%%%%%%%%%%%%%%%%%%%%%%%%%%%%%%%%%%%%%%%%%%%%%%%%%%%%%%%%%%%%%%%%%%%%%%%%%%%%%%%%%
	
	\input{chapters/introduction}
	%!TEX root = ../username.tex
\chapter{Are Virtual Things Real?}
\section{Definitions}
At first glance, the question, an answer to the question "virtual things real?" appears trivial. Clearly, the Pokemon in my handheld game are nothing more than a series of electromagnetic patterns. The same could be said of any sense-experience we have. 
Before we can investigate the claim "are virtual worlds real?", we must first establish what we mean by the claim. In his article \textit{The Virtual and the Real}, David Chalmers describes the term virtual as having several commonplace definitions. One common meaning of the term "virtual", claims the phrase "virtual X" means something along the lines of "as if X but not X" . On that reading, virtual reality is an unreal as-if reality, and virtual reality is no more reality than a virtual kitten is a kitten. \cite{ChalmersVR} While this definition is widely cited in dictionaries and other authoritative sources of definitions, it does not reflect the term’s contemporary definition. The advent of modern computer technology has resulted in a second widely used definition of the term virtual When used in discussions regarding computing, English speakers  tend to mean something along the lines of "a computer-based version of X" when talking about a "virtual X. Under this more contemporary description, virtual reality can be a sort of reality, just a virtual bank is a real bank (insofar as it fulfills all of the roles and functionalities we typically associate with an organization with the label of "bank"). 
For many readers, this recitation of definitions may appear pedantic. However, the dual definitions of the term "virtual" suggests that perhaps the opposition towards virtual realism is in part motivated by linguistics, just as proponents of the identity thesis in the philosophy of mind struggle with the inherent dualist bias in how English speakers and Western Philosophy think about these ideas. I will discuss this view of mine at greater length at later points in this paper, however for now I will merely say the first term should not be applied to descriptions of computer systems that instantiate via a combination of hardware and software an environment in which an agent can interact with entities described by software instructions. To apply the first term to these sorts of systems implies either confusion about the nature of these systems or an attempt to argue against virtual realism by begging the question. 
The second definition presented by Chalmers encompasses a wide array of systems and environments. These spaces range from the permanent and all-encompassing environment depicted in the 1999 science fiction movie \textit{The Matrix} to the ephemeral argumentation of the hit Android and iOS game, \textit{Pokemon Go}. In \textit{Matrix as Metaphysics}, Chalmers argues that "if we are in a Matrix, most of our ordinary beliefs (e.g. that there are tables) are true: if we discovered that we are in a Matrix, instead of saying that there are no tables, we should say instead that tables are computational objects made of bits". Chalmers describes this view as \textit{digitalism}, the claim that (1) Virtual objects really exist and are computational objects;
(2) Events in virtual worlds are largely computational events that really take place. (3) Experiences in virtual reality are non-illusory.
(4) Virtual experiences are as valuable as non-virtual experiences. In contrast to this view is the digital irrealist thesis, which claims that (1) Virtual objects do not really exist. (3) Experiences in virtual reality are illusory. \cite{ChalmersVR}. We shall be using two positions throughout this project as the two primary conflicting views on this topic. However, before we weigh the merits of virtual realism against its opposition, we must first have an account of what is real. This question has been an active battleground for philosophers since the field's inception, 2000 years ago. Most accounts of the real fall either into the dualist, idealist or materialist camps. Substance dualists (SD) hold that there exist two discrete types of substance, physical substance and mental substance.  Materialism is the claim that physical matter is the fundamental substance in the universe. This view is opposed by idealism, the claim that ideas or mental substance is fundamental material of reality. The idealist claims that the universe is purely mental, mentally constructed, or otherwise immaterial in nature. All of these positions are theoretically compatible with either virtual realism or virtual realism. However, each view comes with a series of caveats and ontological commitments which may be a deal breaker for many who hold said view. 




\subsection{Are Virtual Objects real?}
\section{What do we mean by \textit{real}?}
One of the two major schools in Mahayana Buddhism, Yogacara is a form of Buddhistism noted for its denial of the existence of external objects \cite{siderits2007buddhism} . Literally translated, the term Yogacara means "the practice of yoga", this name reflects the school's origins in metaphysical speculation into the nature of yoga and mediation practices. Many advanced mediation practices involve focusing on ones awareness of purely mental entities, the connection between these expertises and the achievement of Enlightenment motivates the Yogacara's idealist understanding of the universe. Yogacaran metaphysics can be characterized by the term \textit{Cittamatra} (English: "consciousness only"), one of the school's other names. Yoagacarans believe that nothing exists besides mental things. This radical form of idealism seems highly counterintuitive and illogical. However, there are many persuasive arguments for this abnormal view. When somebody suffering from cataracts looks at the moon, they have the experience of seeing the moon as if it were covered in hairs. But clearly a hairy moon is no more real than a moon made out of cheese. Yet for the individual suffering from cataracts, their experience of a hairy moon is just as real as the experience of a desolate rocky moon was for the crew of Apollo eleven. So how do we account for what the person is seeing? Yogacaran philosopher Vasubandhu argues that the person with cataracts is aware of a mental image (deemed an impression) that manifests itself as an external object when there is no such thing outside of the mind. This view is motivated by representationalism, the notion that we "what we are directly aware of in waking memory sensory experience is not the external object, but rather a mental image that resembles the object and is caused by sense-object contact" \cite{siderits2007buddhism}.  \newline In contrast to the "impression only" idealism of the Yogacara, the representationalist viewpoint is compatible with the existence of external objects (i.e: a realist standpoint). Vasubandhu argues that the world is nothing but unreal impressions, analogous to the unreal hairs on the moon *seen* by the cataract sufferer.
\newline
Vasubandhu continues by denying the existence of spatial locations. Both realists and idealists like Vasubandhu describe experiences in terms of physical objects, but these experiences could also be explained in terms of images containing colors and shapes. Each of these color/shape images can be described as baring different relations to each other(left, right, etc) \cite{siderits2007buddhism}. This visual change will change over time, but an observer will eventually discover that certain visual features will reoccur periodically in a predicable manner. From these patterns, an agent can construct a phenomenal language that maps onto the all of the visual elements we typically describe in spatial terms. This language may be awkward, but it is the only means to describe these types of experiences in a manner that is amicable to both realists and idealists. This phenomenal language mirrors how computer systems describe and represent entities within a virtual space. Using this language, the realist can object to Vasubandhu's claim with the inter-subjective agreement \cite{siderits2007buddhism}. This argument centered around the fact that, barring special cases in which the experience is solely an impression (i.e: in the hairy moon example) agents seem to have remarkably similar sensory experiences. The realist claims that the different between these special impression-only experiences and *normal* sensory experience is that there is only a inter-subjective agreement in the latter scenario. In other words, the majority of observers are in agreement about the nature of the sensory experience and there are publicly accessible signifiers that can explain the minority's differences in sensory experience of the phenomena in question. From these facts, the realist claims, we can infer that normal sensory experience is independent of the observer's mind and therefore physical objects must exist. Another reply to Yogacaran idealism available to the realist is the argument from efficacy. This counterargument involves comparing sensory experiences which are known to be merely impressions with what are said to be normal sensory experiences. The normal sensory experiences will have clearly observable casual effects on the observer while the impressions have no lasting casual impact on the agent or their surroundings. \newline Vasubandhu counters these realist arguments with the phenomena of the \textit{pretas}, beings cursed to consume urine, blood and other vile things. Vasubandhu explains that karma causes the pretas to perceive an ordinary river as brimming with fowl liquids just as it causes the sinner to perceive the existence of demons and other guardians of hell. The pretas also present a counterexample to the inter-subjective agreement, the impression of a river of filth and demons is not merely the sensory-experience of a single entity, but rather the phenomenal  experience of everyone who has done evil in their past life.  




Having shown shared karmic experience is an adequate substitute for physical substance for explaining the phenomena of intersubjective agreement.  Vasubandhu continues by arguing that spatio-temporal determinacy can be explained without positing the existence of physical objects. Vasubandhu cites dreams as an example of mere impressions the realist might claim exhibit spatio-temporal determinacy and efficacy. Dream entities clearly seem to have a discrete location at a given time within the dream world. Similarly, erotic dreams can have the same sorts of physical effects as an analogous intimate encounter outside of a dream state. \cite{siderits2007buddhism}.
\newline

For many, Vasubandhu's counterexamples will hardly seem sufficient to justify such an outlandish scheme as to propose the nonexistance of physical matter. After all, if we accept a Buddhist concept of karma, we are still left with two competing theories of experience: the impressions only explanation in terms of karmic seeds and the theory of karmic casual laws. \cite{siderits2007buddhism} Vasubandhu employs the \textit{principle of lightness}, a principle from Buddhist philosophy which states that "given two competing theories each of which is equally good at explaining and predicting the relevant phenomena, choose the lighter theory, that is, the theory which posits the least number of unobservable entities" \cite{siderits2007buddhism}. This principle motivates Vasubandhu's move to idealism, why should one posit the existence of material elements when the impressions-only theory has equal explanatory power? Realism introduces unnecessary complexity into our account of existence.  

\section{Virtual Realism and Fictionalism}
Virtual realism is unproblematic for the idealist, who asserts that all of reality is immaterial mental substance (in other words, ideas). However, Idealism is a difficult for some to wrap their heads around. Why ought we accept this view? The classical Western idealist position was first formalized by Bishop George Berkeley. Berkeley's rejection of the existence of physical substance is based upon the claim that we can perceive only our own ideas. Berkeley continues by arguing given that no inert substance (such as matter) can have ideas and since matter is an inert substance has sensible qualities, our understanding of matter is by definition contradictory in nature. Berkeley claims that a denial of the claim "If sensible qualities are perceivable, then they must be ideas" in conjunction with his previous assertion that we can perceive only our own ideas implies a conception of matter that is epistemologically inaccessible and hence meaningless. \cite{berkeley2003a}
\\

 We shall use a similar line of argumentation to justify a virtual realist conception of reality. Virtual objects are weapons, people, buildings and other entities within a virtual environment. These entities are instantiated within the memory of a digital computer by a data structure. Data structures consist of a series of values electromagnetically stored within the memory of the computer. When interpreted by a specific programming language, these values specify the number of bytes to allocate for a given instance or instances of a particular abstract object and specify a series of operations which can be applied to the data stored within the allocated memory. Every action in a given virtual world has a corresponding action within one or more data structure within the memory of the computer which is instancing said virtual world. These data structures can cause the computer to output electromagnetic data which can be transformed by other hardware into images, sounds and other forms of sense data. These electromagnetic emissions also enable it to communicate via wired or wireless connection with other computer systems. Humans and other conscious agents with the appropriate sensory apparatus can perceive these electromagnetic emissions as sense-data.
 \\
 
  Furthermore, data structures themselves (via the hardware they are instantiated within) seem to have some level of causal powers. Computers fly jets, drive cars, perform surgery and drive multi-million user cross-continental worlds like EVE Online and World of Warcraft. \\
  Yet, books, movies and other forms of media also cause real world events. People clearly have strong emotional responses to virtual entities, a computer character or idea from a digital world can invoke power emotions in an agent. Players invest thousands of hours of time \\
  into completing in-game tasks in order to build or collect digital items, abilities or wealth. Indeed, such entities can fetch similar prices to physical items such as cars and can be the subject of theft and immense pride for game players. \cite{EVEArticle}
   

  
  \begin{figure}[ht!]
  	\centering
  	\includegraphics[width=90mm]{4339410221_12d667f5b1_o.png}
  	\caption{A chart detailing physical world market values of digital space ships in EVE Online \label{overflow}}
  	\label{fig:FF7}
  \end{figure}
  	
  The virtual irrealist might claim that these data structures derive their causal power from the programmer who created them or the algorithm which generated them instead of arising from the data structures themselves. Under this view, data structures (and hence virtual objects) are ultimately patterns of interactions between different entities consisting of physical matter whose arrangement is set forth by the actions of a conscious entity. However, the same could be said of books and other other forms of media which the digital realist claims not to constitute real entities (or at least is not committed by definition to claiming). One direction for mitigating this problem is to bite the bullet and adapt a fictionalist stance, which holds that all fictional worlds are real in so much as we can say $\exists X$ where $X$ is some fictional entity within world $Y$. This line of reasoning is resembles modal realism, the position which claims that all possible worlds are as real as the actual world.\cite{lewis1986on} 
 \\
 
 
	 One proponent of this position is David Lewis. Lewis' conception of modal realism can be summarized by the following claims:
\begin{itemize}
	\item Possible worlds exist and they are just as real as our world;
	
	\item Possible worlds are the same sort of things as our world,  differing in content, not in kind;
	\item Possible worlds are irreducible entities in their own right. They cannot be reduced.
\end{itemize}
\cite{lewis2001counterfactuals} 
 For Lewis there is not a single cosmos but rather an infinite multitude of universes. Each of these universes is just as real as our own universe, merely causally separated such that we cannot have empirical knowledge of these universes nor they of our world\cite{lewis1986on} Under this ontology, the term 'actual' describes not whether an entity is real but rather acts as an indexical apparatus for denoting which world an instantiation of an entity belongs to. Thus, the Lewisian modal realist holds that the proposition "I am the actual Spock" is equivalent to the proposition "I am the Spock of this Universe" Propositions of this class "$X$ is the actual $Y$" or "$\exists X$ must be prefixed with the caveat "$X\in$ Universe $N$.
Lewis views propositions about fictional entities containing pretend-assertions.\cite{Lewis1978-LEWTIF} He suggests
that when we say things like "Holmes lived in Baker Street," this should be regarded as
equivalent to the claim that "[i]n the Sherlock Holmes stories, Holmes lived at 221B Baker Street." \cite{Lewis1978-LEWTIF}. Lewis formalizes this notion claiming, "In the fiction world $f$, [entity] $\phi$ is non-vacuously true if and only there exists some 
world where $f$ is told as known fact and $\phi$ is true differs less from our actual world, on
balance, than any world where f is told as known fact and $\Phi$ is not true.
world where $f$ is told as known fact and $\Phi$ is true differs less from our actual world, than any world where $f$ is known to be fact and $\Phi$ is not true." \cite{Lewis}.Lewis argues we must adapt such a conception of fictional events in order to avoid the Frodo's birthday problem. \cite{Lewis1978-LEWTIF} Consider the case of Frodo Baggins, the hobbit protagonist of in J.R.R Tolkien's \textit{Lord of the Rings} fantasy novel series. \begin{itemize}
	\item Frodo is a hobbit. $Sf$
	\item Frodo is hairy. $Hf$
	\item Frodo wore something green at his tenth birthday party. $Gf$
	\item  It is not the case that Frodo wore something green at his tenth birthday party. $\neg Gf$
\end{itemize}
The truth values of the first two propositions can be determined by reading the book. In Tolken's Middle Earth, Tolken affirms that "all hobbits are hairy" and "Frodo is a hobbit". However, we have no way of determining from canon whether Frodo wore something green on his tenth birthday or if he even had a tenth birthday at all since the books never covered such topics. Herein the problem lies, how do we assign a truth value to the propositions $Gf$ and $\neg Gf$, one of these two claims must be false or else we will have a logically contradictory state of affairs. 
	 Lewis proposes several means to account for true statements about fictional things, settling on following two methods: 
 \begin{itemize}
	\item Sentences of the form "In the fiction f, $\Phi$" are non-vacuously true iff if there $\exists$ some
world where $f$ is told as known fact and $\Phi$ is true differs less from our actual world, on
balance, than any world where $f$ is told as known fact and $\Phi$ is not true. \cite{Lewis1978-LEWTIF}
\item Sentences of the form "In the fiction $f$, $\Phi$" is non-vacuously true iff,
whenever $w$ is one of the collective belief worlds of the community of origin for $f$, then
some world where f is told as known fact and $\Phi$ is true differs less from the world $w$, on
balance, than does any world where $f$ is told as known fact and $\Phi$ is not true." \cite{Lewis1978-LEWTIF}
\end{itemize}
Lewis' two proposals for how to account for truth in fiction fail to address the worry raised by the Frodo's birthday problem since the idea of Frodo wearing a blue shirt is not so extraordinary either in Middle Earth or in our Universe as to render the proposition ($Gf$) or its negation implausible. However, a Lewisian conception of truth in fiction does render false more bizarre non-canonical claims such as the claim that Frodo has an extra nose on his chest or Frodo owned a nightclub in the Shine prior to the events of the \textit{Lord of the Rings}. Lewis' understanding of truth within fictional worlds also protects the power of analytic truths to provide answers to questions like "Does Frodo have a birthday?". Further, this 
\\
Up until this point, we have discussed the epistemological concerns associated with possible worlds. However, we yet to ask what \textit{are} virtual worlds? We have seen above a physicalist of virtual worlds. This view conceive of entities within virtual worlds as the electromagnetic patterns running within a digital computer's hardware. 
Although not the explicit position of David Lewis with respect to we ought to account for fictional worlds metaphysically, one possible reading of Lewisian modal realism  since there could be logically possible universe corresponding exactly to universe of Middle Earth as depicted in J.R.R Tolkien's \textit{Lord of the Rings} fantasy novel series, we have true propositions such as ($\forall $entities $\in$ Middle Earth)($\exists$ entities). Lewis' modal realist thesis allows for infinitely many possible worlds in which we can justified true beliefs about entities that seem fictional to us in our universe. Such a move might seem at first glance to be an acceptable means of 
accommodate a fictionalist worldview. However, Lewis clearly states that possible worlds have causal closure. Using Lewis' own account of how to evaluate truth-value when dealing with fictional statements does little either to account for the causal powers of fictional things.  



 

  \section{Awareness and Virtual Reality}
A major factor when discussing the realism of a virtual environment is the agent or user's knowledge about said system. A user's sense of reality is highly dependent on their past experiences and subjective conception of reality. Just as the prisoners in Plato's  allegory of the cave believe the shadows on the wall to be real, an agent who was born and raised experiencing only the sense-data of digital environment like the one depicted in the movie \textit{The Matrix} will view sense-data from this machine as real, regardless of how unnatural or unreal it might appear to a human from "real world". Knowledge of whether one is a "brain in a vat" or within the "real world" is a major factor in determine of whether one doubts the reality of their experience. 
 \subsection{Materialism and VR}
Materialism holds that only finite physical substance embodies reality. For the materialist to accept virtual realism, they must adapt a strictly empiricist epistemological view, claiming sense-data is the sole source of justified true beliefs about the universe. For a materialist with this view, whatever they experience as the world around them is the world around them. A materialist born and raised in the Matrix would have no means (barring glitches in the Matrix) of deducing that she is experiencing sense-data that has been fed to her by a complex computer system. However, this view also is friendly to claims of virtual realism about contemporary VR technology. The output from a VR system is sense-data too. Interactions with objects in VR still have the ability to cause mental states in an agent, just like any other material entity.

 
\chapter{Quantifying the realness of virtual worlds }\label{text}
\section{Introduction}


Computer graphics have progressed immensely since the dawn of 3d computer graphics. Modern day computers are capable of rendering 3d models with with millions of polygons and employ complex shaders (computer programs that determine which colors and textures are used to draw the model) and other effects to create life-like images. However, realism in computer graphics is not merely a measure of the amount number of polygons or complexity of the shaders, physics engine or lighting effects. Small details like the uniqueness of terrain and objects, a lack of diversity in texture pattern and quality and other factors can have a significant impact on whether an environment appears realistic to a user. In this section, we shall survey and critique contemporary methods for quantitatively measuring realism in 3d computer worlds and posit our own measurement scheme. 
\\

The term "realistic" compasses a wide array of subjective and objective qualities. One key factor in determining whether an experience in a 3d environment is realistic for a user is immersion. Authors in the field of digital communication and human-computer factors use the term “immersion” to describe a wide array of characteristics a digital space might have. Many authors use the term immersion to measure the ability of the system to "shut out:" or "pull" the user out of their awareness of physical reality. 
Slater and Wilbur (1997) note that an environment is more likely to feel immersive if it:
a) offers high fidelity simulations
through multiple sensory modalities, b) finely maps a user’s virtual bodily
actions to their physical body's counterparts, and c) removes the participant
from the external world through self-contained plots and narratives. 
These factors point to importance of sense of presence in making an immersive experience. However, other authors attempt to measure i through realistic graphics. Some researchers in this field attempt to quantify realism by taking snapshots of virtual environments and measuring aspects of the resulting images such as gradient variance, color variance and shadow
softness \cite{Wang:2011:RRP:2013879.2014089}. Others leverage qualitative data collected from user interviews to compare levels of realism between different virtual worlds and across hardware platforms.
\\

 In this paper, we shall use the latter approach to measuring realness of virtual worlds. Our argument for the superiority of this methodology hinges on the fact that static screen-shots fail to account for the impact of animation, field of view, audio effects, type of control peripherals and other key aspects of user-experience on the immersiveness of an experience. A virtual environment's immersiveness is not merely a product of its graphics engine, physics engine and other software components but rather a composite of the proper application of software techniques and game design paired with a hardware system that well complements these software factors. As a result, it is improper to discuss the reality or immersive properties of
a piece of software independently of the hardware platform it is running on top of. 

\section{Emotion and immersion}
As described in introduction to this section, the realism of a virtual environment is a holistic measurement of the immersiveness of the player's experience within the game world. Games with inferior graphics systems but which feature a highly captivating story or game play mechanic are easier to lose oneself within then a bland game with state-of-the-art graphics. Central to a game or other form of simulation's ability to captivate a user is its capacity to invoke an emotional response in a user or player. Games with primitive graphics and game mechanics can still have powerful emotive power, Aerith's death in the 1997 JRPG (Japanese RPG) Final Fantasy VII still prompts strong emotional reactions, despite its primitive character models and linear gameplay in comparison with modern day games in the same genre. \cite{fig:FF7} 
\begin{figure}[ht!]
\centering
\includegraphics[width=90mm]{maxresdefault.jpg}
\caption{Final Fantasy 7 Aerith Death Cutscene \label{overflow}}
\label{fig:FF7}
\end{figure}
Clearly, the ability to invoke emotion is not something that can be measured using surface roughness or other such technical metrics. Storytelling is an essential aspect in creating an immersive experience. However, there are many games without plots that also offer a highly immersive experience for players. One such game is Minecraft, which offers players a 3d sandbox environment in which they must survive in a hostile world while building and exploring a procedurally generated voxel world containing caves mountains and countless other environments. This game features relatively primitive shaders and lighting effects and has a very minimal plot and few NPCs, yet the game still offers a highly immersive and addictive player experience. Players denote thousands of hours to building detailed and complex structures within the game world and express genuine stress and unease when faced with the in-game monsters, despite their unrealistic appearances. \cite{fig:MCmonster} 
\begin{figure}[ht!]
\centering
\includegraphics[width=90mm]{minecraft-32.jpg}
\caption{Minecraft Gameplay \label{overflow}}
\label{fig:Minecraft}
\end{figure}
\cite{fig:MCmonster} 
\begin{figure}[ht!]
\centering
\includegraphics[width=90mm]{minecraft-spider-300x240.jpg}
\caption{Minecraft Hostile NPCs (source: http://www.minecraftlover.com) \label{overflow}}
\label{fig:MCmonster}
\end{figure}
A game's capacity to elicit an emotional response in the user is an essential and undeniable factor in its immersive qualities. We see this in \cite{Brown04agrounded}, where the authors found that a game's
\subsection{What Are Immersive Qualities }

	\input{chapters/chapter2}
	\input{chapters/chapter3}
	%\input{chapters/chapter4}
	%\input{chapters/chapter5}
	%\input{chapters/chapter6}
	%\input{chapters/chapter7}
	%\input{conclusion}
	
	%%%%%%%%%%%%%%%%%%%%%%%%%%%%%%%%%%%%%%%%%%%%%%%%%%%%%%%%%%%%%%%%%%%%%%%%%%%%%%%%%%%%%%%%%%%%%%
	% This section starts the back matter. The back matter includes appendices, indicies, and the %bibliography
	%%%%%%%%%%%%%%%%%%%%%%%%%%%%%%%%%%%%%%%%%%%%%%%%%%%%%%%%%%%%%%%%%%%%%%%%%%%%%%%%%%%%%%%%%%%%%%
	
	\backmatter
	
	%\input{appendices/math}
	%\input{appendices/java}
	%\input{appendices/cpp}
	%\input{appendices/afterword}
	
	%%%%%%%%%%%%%%%%%%%%%%%%%%%%%%%%%%%%%%%%%%%%%%%%%%%%%%%%%%%%%%%%%%%%%%%%%%%%%%%%%%%%%%%%%%%%%%
	% This section would be used if you are not using BibTeX. Look at Kopka and Daly for how to
	% format a bibliography manually as well as how to use BibTeX.
	%%%%%%%%%%%%%%%%%%%%%%%%%%%%%%%%%%%%%%%%%%%%%%%%%%%%%%%%%%%%%%%%%%%%%%%%%%%%%%%%%%%%%%%%%%%%%%
	
	%\begin{thebibliography}{99}
	%\bibitem{}
	%\bibitem{}
	%\end{thebibliography}
	
	%%%%%%%%%%%%%%%%%%%%%%%%%%%%%%%%%%%%%%%%%%%%%%%%%%%%%%%%%%%%%%%%%%%%%%%%%%%%%%%%%%%%%%%%%%%%%%
	% We used BibTeX to generate a Bibliography. I would recommend this method. However, it is %not required.
	%%%%%%%%%%%%%%%%%%%%%%%%%%%%%%%%%%%%%%%%%%%%%%%%%%%%%%%%%%%%%%%%%%%%%%%%%%%%%%%%%%%%%%%%%%%%%%
	
	\renewcommand\bibname{References} % changes the name of the Bibliography
	
	\nocite{*} % This command forces all the bibliography references to be printed -- not just 
	              % those that were explicitly cited in the text.  If you comment this out, the bibliography
	              % will only include cited references.
	\ifthenelse{\boolean{woosterchicago}}{
	\bibliographystyle{woosterchicago}}{\ifthenelse{\boolean{achemso}}{
	\bibliographystyle{achemso}}{\bibliographystyle{plainnat}}} % if you have used the woosterchicago class option then your references and citations will be in Chicago format. If you have used the achemso class option then your references and citations will be in the American Chemical Society format. If you do not specify a citation format then the default Wooster format will be used.
	\bibliography{references} % load our Bibliography file
	
	%%%%%%%%%%%%%%%%%%%%%%%%%%%%%%%%%%%%%%%%%%%%%%%%%%%%%%%%%%%%%%%%%%%%%%%%%%%%%%%%%%%%%%%%%%%%%%
	%
	%                                                                Index
	%
	%  Uncomment the lines below to include an index. To get an index you must put 
	%  \index{index text} after any words that you want to appear in the index.
	%  Subentries are entered as \index{index text!subentry text}. You must also run the
	%  makeindex program to generate the index files that LaTeX uses. The PCs are set to run
	%  makeindex automatically.
	%
	%%%%%%%%%%%%%%%%%%%%%%%%%%%%%%%%%%%%%%%%%%%%%%%%%%%%%%%%%%%%%%%%%%%%%%%%%%%%%%%%%%%%%%%%%%%%%%
	
	\ifthenelse{\boolean{index}}{
	\cleardoublepage
	\phantomsection
	\addcontentsline{toc}{chapter}{Index}
	\printindex}{}
	
	%%%%%%%%%%%%%%%%%%%%%%%%%%%%%%%%%%%%%%%%%%%%%%%%%%%%%%%%%%%%%%%%%%%%%%%%%%%%%%%%%%%%%%%%%%%%%%
	%
	%                                                                Colophon
	%
	%  A Colophon is a section of a printed document that acknowledges the designers and printers of the work.
	% The colophon also includes information about the fonts and paper used in the printing. It is not required 
	% for your IS and can be commented out.
	%
	%%%%%%%%%%%%%%%%%%%%%%%%%%%%%%%%%%%%%%%%%%%%%%%%%%%%%%%%%%%%%%%%%%%%%%%%%%%%%%%%%%%%%%%%%%%%%%
	
	\ifthenelse{\boolean{colophon}}{
	\begin{colophon}
	This Independent Study was designed by Dr. Jon Breitenbucher.\newline
	It was edited and set into type in Wooster, Ohio,\newline
	using the \ifthenelse{\boolean{xetex}}{\XeTeX\ typesetting system designed by Jonathan Kew}{\LaTeX\ typesetting system designed by Leslie Lamport}\newline
	and based on the original \TeX\ system of Donald Knuth.\newline
	It was printed and bound by Office Services at The College of Wooster.
	
	The text face is Adobe Garamond Pro, designed by Robert Slimbach.\newline
	This is the Opentype version distributed by Adobe Systems\newline
	and purchased as part of the Adobe Type Classics for Learning.
	
	The paper is standard laser copier paper and not of archival quality.
	\end{colophon}}{}
	\clearpage\thispagestyle{empty}\null\clearpage
	\end{document}